\documentclass{beamer}

\usepackage[utf8]{inputenc}
\usepackage[danish]{babel}
\graphicspath{{../imgs/}}

\usetheme{Ilmenau}
\usecolortheme{beaver}
\uselanguage{Danish}
\languagepath{Danish}

\title{DaLUKE}
\subtitle{
    Den Dansktalende Sprogmodel MED VIDEN
}
\author[Søren Holm, Asger Schultz]{Søren Winkel Holm, Asger Laurits Schultz}
\institute[DTU]{Danmarks Tekniske Universitet}
\date{7. juli 2021}

\begin{document}

\begin{frame}
    \titlepage
\end{frame}

%Kom hurtigt til resultater
%Relativt få slides
%Vis arbejde
%Understøt med figurer
%Evt demonstrerer med karaktereksempel
%Kort om anvendelse - meget efterspurgt i industrien
%Kom hurtigt frem til resultater
%Hav repræsentationen med

% Asger: Hvad er problemet? og Hvordan gik det?
% Søren: Hvordan vil vi løse det? og Hvad har vi lært?

\begin{frame}
    \frametitle{Fremlæggelsen}
    \footnotesize
    \tableofcontents
\end{frame}

\section{Hvad er problemet?}
\begin{frame}
    % 1
    \frametitle{Sprogforståelse mangler viden og generaliseringsevne til lavressourcesprog}
\end{frame}

\begin{frame}
    % 1
    \frametitle{Navngiven entitetsgenkendelse (NER)}
\end{frame}

\section{Hvordan vil vi løse det?}
\begin{frame}
    % 2
    \frametitle{Hvad gør Lukas anderledes?}
\end{frame}

\begin{frame}
    % 3
    \frametitle{Prætræning af DaLUKE}
\end{frame}

\section{Hvordan gik det?}
\begin{frame}
    % 3
    \frametitle{NER på dansk}
\end{frame}

\begin{frame}
    % 3
    \frametitle{Entitetsbevidst selvopmærksomhed}
\end{frame}

\begin{frame}
    % 3
    \frametitle{Transferlæring}
\end{frame}

\begin{frame}
    % 2
    \frametitle{Datasætmodifikationer}
\end{frame}

\section{Hvad har vi lært?}

\begin{frame}
    % 2
    \frametitle{NER er tilfældigt}
    % Lille sample
    % RNG
    % Krydsvalidering
\end{frame}

\begin{frame}
    % 3
    \frametitle{Repræsentationer}
\end{frame}

\begin{frame}
    % 3
    \frametitle{Forudsigelseseksempler}
\end{frame}

\begin{frame}
    % 1
    \frametitle{Afslutning med maskot}
\end{frame}

\end{document}

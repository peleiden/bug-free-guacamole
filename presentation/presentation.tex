\documentclass{beamer}

\usepackage[utf8]{inputenc}
\usepackage[danish]{babel}

\usetheme{Ilmenau}
\usecolortheme{beaver}
\uselanguage{Danish}
\languagepath{Danish}

\title{DaLUKE}
\subtitle{
    Den Dansktalende Sprogmodel MED VIDEN
}
\author[Søren Holm, Asger Schultz]{Søren Winkel Holm, Asger Laurits Schultz}
\institute[DTU]{Danmarks Tekniske Universitet}
\date{7. juli 2021}

\begin{document}

\begin{frame}
    \titlepage
\end{frame}

%Kom hurtigt til resultater
%Relativt få slides
%Vis arbejde
%Understøt med figurer
%Evt demonstrerer med karaktereksempel
%Kort om anvendelse - meget efterspurgt i industrien
%Kom hurtigt frem til resultater
%Hav repræsentationen med

\begin{frame}
    \frametitle{Fremlæggelsen}
    \footnotesize
    \tableofcontents
\end{frame}

\section{Introduktion}
\subsection{Hvorfor er problemet interessant?}
\subsection{Hvad har andre gjort?}

\section{Teori og metode}
\subsection{NLP: Nederen, langvarig pine}
\subsection{Transformeren og Lukas}
\subsection{Datasæt}
\subsection{Vores modifikationer}

\section{Resultater og eksperimenter}
\subsection{Reproduktion kortere end mænd med små hænder}
\subsection{Hovedresultat}
\subsection{Étslidesgennemgang af resultater}

\section{Diskussion}
\subsection{Validitet af resultater: Kan DaNLP finde ud af statistik?}
\subsection{Hvad har vi lært? Ingenting}
\subsection{Videre arbejde}

\begin{frame}
    \frametitle{Lyt til os}
    Her følger de vise ord
    \begin{definition}
        Sprog er en social konstruktion
    \end{definition}
    \begin{example}
        ipsum dolor
    \end{example}
    \begin{block}{Hej K}
        Big K is big
    \end{block}
    \begin{itemize}
        \pause
        \item Hej
        \pause
        \item med
        \pause
        \item dig
    \end{itemize}
\end{frame}

\end{document}

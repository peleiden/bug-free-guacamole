% !BIB TS-program = biber
\documentclass[12pt, fleqn]{report}

\usepackage[utf8]{inputenc}
\usepackage[english]{babel}

\PassOptionsToPackage{hyphens}{url}\usepackage{hyperref}
\usepackage[top=2.5cm, bottom=2.5cm, left=3cm, right=3cm, includeheadfoot]{geometry}
\usepackage{fancyhdr}
\usepackage{graphicx}
\usepackage{float}
\usepackage{changepage}
\usepackage[nottoc, numbib]{tocbibind}
\usepackage{lastpage}
\usepackage{setspace}
\usepackage[bottom]{footmisc}

\usepackage{amsmath}
\usepackage{amssymb}
\usepackage{nicefrac}
\usepackage{icomma}

\usepackage{color} %red, green, blue, yellow, cyan, magenta, black, white
\usepackage[dvipsnames]{xcolor}
\usepackage{titlesec}
\usepackage{listings}

\usepackage{textcomp} % To avoid annoying \perthousand, \micro warnings
\usepackage{subfiles}
\usepackage{csquotes} % To avoid biber warnings
\usepackage[backend=biber, style=alphabetic, citestyle=alphabetic, maxcitenames=4, maxbibnames=4, mincitenames=2]{biblatex}


\allowdisplaybreaks
% Nicer chapters
\titleformat{\chapter}{\normalfont\huge}{\thechapter.}{20pt}{\huge}
\titlespacing*{\chapter}{0pt}{0pt}{40pt}

\fancypagestyle{plain}
{
    \fancyhf{}
    \rfoot{Page \thepage{}~of  \pageref{LastPage}}
    \renewcommand{\headrulewidth}{0pt}
}
\pagestyle{fancy}
\fancyhf{}

\numberwithin{equation}{chapter}
\numberwithin{footnote}{chapter}
\numberwithin{figure}{chapter}
\numberwithin{table}{chapter}

\lstset{frame=None,
    language={python},
    inputencoding=ansinew,
    literate=
    {æ}{{\ae}}1
    {å}{{\aa}}1
    {ø}{{\o}}1
    {Æ}{{\AE}}1
    {Å}{{\AA}}1
    {Ø}{{\O}}1,
    aboveskip=3mm,
    belowskip=3mm,
    showstringspaces=false,
    columns=flexible,
    basicstyle={\footnotesize\ttfamily},
    numbers=left,
    numberstyle=\tiny\color{gray}\ttfamily,
    keywordstyle=\color{blue}\ttfamily,
    ndkeywordstyle=\color{blue}\ttfamily,
    commentstyle=\color{gray}\ttfamily,
    stringstyle=\color{OliveGreen}\ttfamily,
    breaklines=true,
    breakatwhitespace=true,
    tabsize=4,
    escapeinside={<@}{@>},
    lineskip={-1.5pt},
    xleftmargin=1cm,
    xrightmargin=1cm
}

% Helper commands
\newcommand{\code}[1]{{\texttt{\small#1}}}
\newcommand{\numberthis}{\addtocounter{equation}{1}\tag{\theequation}}
\newcommand{\acomm}[1]{\hspace{2.5cm}\text{#1}}
\newcommand{\low}[1]{\ensuremath{_\textup{#1}}}

\newcommand{\andim}{\textup{ and }}
\newcommand{\raq}{\Rightarrow\quad}
\newcommand{\lraq}{\Leftrightarrow\quad}
\newcommand{\qandq}{\quad\wedge\quad}
\newcommand{\qorq}{\quad\vee\quad}
\newcommand{\diff}[2]{\ensuremath{\frac{\md #1}{\md #2}}}
\newcommand{\md}{\ensuremath{\text{d}}}

\newcommand{\ctp}[1]{\ensuremath{\cdot10^{#1}}}
\newcommand{\reci}{\ensuremath{^{-1}}}
\newcommand{\twopow}{\ensuremath{^{2}}}
\newcommand{\re}[1]{\ensuremath{^{#1}}}

\newcommand{\me}{\ensuremath{\operatorname{e}}}
\newcommand{\eul}[1]{\ensuremath{\me^{#1}}}
\newcommand{\len}[1]{\ensuremath{\left\lvert#1\right\rvert}}
\newcommand{\half}{\ensuremath{\frac{1}{2}}}
\newcommand{\third}{\ensuremath{\frac{1}{3}}}
\newcommand{\fourth}{\ensuremath{\frac{1}{4}}}
\newcommand{\transpose}[1]{\ensuremath{#1^{\textup T}}}

\newcommand{\NN}{\ensuremath{\mathbb N}}
\newcommand{\ZZ}{\ensuremath{\mathbb Z}}
\newcommand{\QQ}{\ensuremath{\mathbb Q}}
\newcommand{\RR}{\ensuremath{\mathbb R}}
\newcommand{\CC}{\ensuremath{\mathbb C}}
\newcommand{\LL}{\ensuremath{\mathbb L}}
\newcommand{\PP}{\ensuremath{\mathbb P}}

\newcommand{\unit}[1]{\ensuremath{\:\text{#1}}}
\newcommand{\pro}{\ensuremath{\unit{\%{}}}}


\addbibresource{references.bib}

\chead{}
\rhead{Technical University of Denmark}
\rfoot{Page \thepage{}~of \pageref{LastPage}}

\graphicspath{{imgs/}{../imgs/}}
\linespread{1.15}

\begin{document}

%https://en.wikibooks.org/wiki/LaTeX/Title_Creation
\begin{titlepage}
    \centering
    {\scshape\LARGE Technical University of Denmark \par}
    \vspace{1cm}
    {\scshape\Large Bachelor project in Artificial Intelligence and Data\par}
    \vspace{1.5cm}
    {\huge\bfseries daLUKE: The Entity-Aware, Danish Language Model\par}
    \vspace{2cm}
    \begin{large}
        \centering
        \begin{tabular}{ccc}
                Søren Winkel Holm & Asger Laurits Schultz\\
                \code{s183911@student.dtu.dk} & \code{s183912@student.dtu.dk}
        \end{tabular}
    \end{large}\par
    \vfill
    Project: \textit{An Open Source Danish Knowledge Graph Language Model}\par
    Software repository
    \code{\href{https://github.com/peleiden/daLUKE}{github.com/peleiden/daLUKE}}\par
    Demonstration:
    \code{\href{https://peleiden.github.io/daLUKE}{peleiden.github.io/daLUKE}}\par
    \vspace{2cm}
    Supervisors:\par
    Lars Kai Hansen, DTU Compute\\
    Michael Riis Andersen, DTU Compute\\
    Victor Elkjær Birk, IBM Services
    \vfill
    {\large \today\par}
\end{titlepage}
\begin{abstract}
    Previous Danish Named Entity Recognition (NER) Results reproduced.
    The English-speaking NER performance of LUKE \cite{yamada2020luke} reproduced.
    A Danish version of LUKE pretrained on Danish text corpus presented as \code{daLUKE} with NER as a benchmark.
\end{abstract}

\setcounter{tocdepth}{1}
\tableofcontents
\setcounter{tocdepth}{2}
\subfile{sections/intro}

\subfile{sections/data}

\subfile{sections/methods}

\subfile{sections/results}

\subfile{sections/discussion}

% \subfile{sections/conclusion}

\printbibliography[heading=bibintoc]

% \appendix
% \subfile{sections/appendix}
\end{document}

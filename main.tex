% !BIB TS-program = biber
\documentclass[12pt, fleqn]{report}

\input{praeambulum.tex}

\addbibresource{references.bib}

\chead{}
\rhead{Technical University of Denmark}
\rfoot{Page \thepage{}~of \pageref{LastPage}}

\graphicspath{{imgs/}{../imgs/}}
\linespread{1.15}

\begin{document}

%https://en.wikibooks.org/wiki/LaTeX/Title_Creation
\begin{titlepage}
    \centering
    {\scshape\LARGE Technical University of Denmark \par}
    \vspace{1cm}
    {\scshape\Large Bachelor project in Artificial Intelligence and Data\par}
    \vspace{1.5cm}
    {\huge\bfseries daLUKE: The Entity-Aware, Danish Language Model\par}
    \vspace{2cm}
    \begin{large}
        \centering
        \begin{tabular}{ccc}
                Søren Winkel Holm & Asger Laurits Schultz\\
                \code{s183911@student.dtu.dk} & \code{s183912@student.dtu.dk}
        \end{tabular}
    \end{large}\par
    \vfill
    Project: \textit{An Open Source Danish Knowledge Graph Language Model}\par
    Software repository
    \code{\href{https://github.com/peleiden/daLUKE}{github.com/peleiden/daLUKE}}\par
    Demonstration:
    \code{\href{https://peleiden.github.io/daLUKE}{peleiden.github.io/daLUKE}}\par
    \vspace{2cm}
    Supervisors:\par
    Lars Kai Hansen, DTU Compute\\
    Michael Riis Andersen, DTU Compute\\
    Victor Elkjær Birk, IBM Services
    \vfill
    {\large \today\par}
\end{titlepage}
\begin{abstract}
    Previous Danish Named Entity Recognition (NER) Results reproduced.
    The English-speaking NER performance of LUKE \cite{yamada2020luke} reproduced.
    A Danish version of LUKE pretrained on Danish text corpus presented as \code{daLUKE} with NER as a benchmark.
\end{abstract}

\setcounter{tocdepth}{1}
\tableofcontents
\setcounter{tocdepth}{2}
\subfile{sections/intro}

\subfile{sections/data}

\subfile{sections/methods}

\subfile{sections/results}

\subfile{sections/discussion}

% \subfile{sections/conclusion}

\printbibliography[heading=bibintoc]

% \appendix
% \subfile{sections/appendix}
\end{document}
